\documentclass{beamer}
\usepackage{listings}
\lstset{
%language=C,
frame=single, 
breaklines=true,
columns=fullflexible
}
\usepackage{subcaption}
\usepackage{setspace}
\usepackage{url}
\usepackage{tikz}
\usepackage{tkz-euclide} % loads  TikZ and tkz-base
%\usetkzobj{all}
\usepackage[utf8]{inputenc}
\usepackage{longtable}
\usetikzlibrary{calc,math}
\usepackage{float}
\newcommand\norm[1]{\left\lVert#1\right\rVert}
\renewcommand{\vec}[1]{\mathbf{#1}}
\usepackage[export]{adjustbox}
\usepackage[utf8]{inputenc}
\usepackage{amsmath}
\usetheme{Boadilla}
\newcommand\mytextbullet{\leavevmode%
\usebeamertemplate{itemize item}\hspace{.5em}}

\bibliographystyle{IEEEtran}

\usepackage{color}

\title{Navigation and Communication for UGV/UAV}
\author{Sachinkumar Omprakash Dubey}
\institute{Indian Institute of Technology, Hyderabad.}
\date{\today}


\begin{document}


\begin{frame}
\titlepage
\end{frame}
\section{Content}
\begin{frame}
\frametitle{Content}
\begin{columns}
\column{1\textwidth}
  \begin{itemize}
  \item Introduction
  \item Motor control basic
  \item Serial communication protocols
  \item ESP32 based applications
  \item Vaman based applications
  \item SATCOM for UAV communication
  \item UGV control using NB-IoT setup
  \end{itemize}
\end{columns}

\end{frame}


\section{Introduction}
\begin{frame}
\frametitle{Introduction}
\begin{columns}
\column{1\textwidth}
  \begin{itemize}
  \item This thesis titled "Navigation and communication for UGV/UAV, consists of two parts:
   \begin{itemize}
  \item The navigation part of this thesis includes utilization of various controllers to implement applications on UGV/UAV hardware.
  \item The communication part explores the use of SATCOM and NBIoT for communicating with UAV.
  \end{itemize}
  \item UGV and UAV kits are ideal low-cost prototype systems for testing software before scaling it up and putting it on a real ground vehicle or a more complicated UAV system. 
  \end{itemize}
\end{columns}
\end{frame}

\section{UGV kit hardware}
\begin{frame}
\frametitle{UGV kit hardware}
\begin{figure}[h!]
  \centering
  \includegraphics[width=0.8\linewidth]{./figs/UGV_components_1.png}
  \caption{UGV kit hardware}
  \label{UGV_kit_hardware}
\end{figure}
%\begin{columns}
%\column{1\textwidth}
%\end{columns}
\end{frame}

\section{UAV kit hardware}
\begin{frame}
\frametitle{UAV kit hardware}
\begin{figure}[h!]
  \centering
  \includegraphics[width=0.9\linewidth]{./figs/UAV_components.png}
  \caption{UAV kit hardware}
  \label{fig:side3}
\end{figure}
%\begin{columns}
%\column{1\textwidth}
%\end{columns}
\end{frame}

\section{Controllers}
\begin{frame}
\frametitle{Controllers}
\begin{table}[H]
\centering
\resizebox{\textwidth}{!}{
\begin{tabular}{|l|c|c|c|}
\hline
\textbf{Parameters} & \textbf{Arduino Uno} & \textbf{Raspberry Pi 3B} & \textbf{ESP-32} \\ \hline
\textbf{Processor} & ATMega328P & \begin{tabular}[c]{@{}c@{}}Quad-core Broadcom \\ BCM2837 (4×Cortex-A53)\end{tabular} & \begin{tabular}[c]{@{}c@{}}Xtensa Dual-Core 32-bit \\ LX6 with 600 DMIPS\end{tabular} \\ \hline
\textbf{GPU} & - & \begin{tabular}[c]{@{}c@{}}Broadcom VideoCore IV \\ @ 250 MHz\end{tabular} & - \\ \hline
\textbf{Operating voltage} & 5V & 5V & 3.3V \\ \hline
\textbf{Clock speed} & 16 MHz & 1.2GHz & 26 MHz – 52 MHz \\ \hline
\textbf{System memory} & 2kB & 1 GB & \textless{}45kB \\ \hline
\textbf{Flash memory} & 32 kB & - & up to 128MB \\ \hline
\textbf{EEPROM} & 1 kB & - & - \\ \hline
\textbf{\begin{tabular}[c]{@{}l@{}}Communication \\ supported\end{tabular}} & \begin{tabular}[c]{@{}c@{}}IEEE 802.11 b/g/n\\ Bluetooth via Shield\end{tabular} & \begin{tabular}[c]{@{}c@{}}IEEE 802.11 b/g/n\\ Bluetooth, Ethernet Serial\end{tabular} & IEEE 802.11 b/g/n \\ \hline
\textbf{\begin{tabular}[c]{@{}l@{}}Development \\ environments\end{tabular}} & Arduino IDE & \begin{tabular}[c]{@{}c@{}}Any linux \\ compatible IDE\end{tabular} & Arduino IDE, Lua Loader \\ \hline
\textbf{\begin{tabular}[c]{@{}l@{}}Programming \\ language\end{tabular}} & Embedded C, C++ & \begin{tabular}[c]{@{}c@{}}Python, C, C++, Java,\\ Scratch, Ruby\end{tabular} & Embedded C, C++ \\ \hline
\textbf{I/O Connectivity} & SPI I2C UART GPIO & \begin{tabular}[c]{@{}c@{}}SPI DSI UART \\ SDIOCSI GPIO\end{tabular} & UART, GPIO \\ \hline
\end{tabular}
}
\caption{Comparison between Arduino Uno, Raspberry Pi 3B and ESP-32}
\end{table}
\end{frame}

\section{Controllers (Vaman)}
\begin{frame}
\frametitle{Controllers (Vaman)}
\begin{columns}
	\column{0.5\textwidth}
	\begin{itemize}
		\item On-board dual processor (ARM + FPGA)
		\item On-board WiFi/BT/BLE connectivity with ESP32
		\item $\mu$SD card support
		\item On-board inertial measurement unit
		\item On-board BMO055 smart fusion sensor
		\item On-board DPS310 provides pressure, humidity and temperature monitoring
	\end{itemize}

	\column{0.5\textwidth}
	\begin{figure}[h!]
  		\centering
  		\includegraphics[width=0.8\linewidth]{./figs/Vaman.png}
  		\caption{Vaman - Pygmy BB4}
  		\label{Vaman}
	\end{figure}
\end{columns}
\end{frame}

\section{Motor control using PWM}
\begin{frame}
\frametitle{Motor control using PWM}
\begin{itemize}
	\item A pulse width modulation speed control system works by sending a series of "ON-OFF" pulses
to the motor. The frequency of square wave is kept constant while varying the duty cycle (the
fraction of time that the output voltage is "ON" compared to when it is "OFF").
	\item By changing the width of the ON duration, one can control the average DC voltage applied to
the motor. The below equation (\ref{eq1}) gives the relation between the Duty cycle ($D$) and the average voltage:

\begin{align}
V_{dc}&={\frac {1}{T}}\int _{0}^{T}v_{PWM}(t)\,dt  \label{eq1} \\
V_{dc} &= {\frac {1}{T}}\left(\int _{0}^{DT}v_{\text{max}}\,dt+\int _{DT}^{T}v_{\text{min}}\,dt\right) \nonumber \\ \nonumber
&={\frac {1}{T}}\left(D\cdot T\cdot v_{\text{max}}+T\left(1-D\right)v_{\text{min}}\right)\\ \nonumber
&=D\cdot v_{\text{max}}+\left(1-D\right)v_{\text{min}} 
\end{align}
\end{itemize}
\end{frame}

\section{Motor control using PWM 1}
\begin{frame}
\frametitle{Motor control using PWM \small{(Continued)}}
\begin{figure}[t!]
    \centering
    \begin{subfigure}[t]{0.5\textwidth}
        \centering
        \includegraphics[width = 7cm]{./figs/PWM_speed_control.png}
        \caption{PWM speed control}
    \end{subfigure}%
    ~ 
    \begin{subfigure}[t]{0.5\textwidth}
        \centering
        \includegraphics[width = 4.5cm]{./figs/Motor_driver_L298.jpg}
        \caption{Dual motor driver module (L298N)}
    \end{subfigure}

\end{figure} 	
\end{frame}




\section{ESP32 Based Applications 1}
\begin{frame}
\frametitle{ESP32 Based Applications-1}
\begin{columns}
	\column{0.9\textwidth}
	\begin{figure}[h!]
  		\centering
  		\includegraphics[width=\linewidth]{./figs/Wiring_UGV_flysky.png}
  		\caption{UGV Navigation using Fly-sky transmitter \& receiver (ESP32)}
  		\label{Wiring_UGV_flysky}
	\end{figure}
\end{columns}
\end{frame}

\section{ESP32 Based Applications 2}
\begin{frame}
\frametitle{ESP32 Based Applications-2}
\begin{columns}
	\column{0.8\textwidth}
	\begin{figure}[h!]
  		\centering
  		\includegraphics[width=\linewidth]{./figs/Wiring_UGV_speech.png}
  		\caption{UGV Navigation using Android phone (ESP32)(Manual and Speech)}
  		\label{Wiring_UGV_speech}
	\end{figure}
\end{columns}
\end{frame}

\section{ESP32 Based Applications 3}
\begin{frame}
\frametitle{ESP32 Based Applications-3}
\begin{columns}
	\column{0.95\textwidth}
	\begin{figure}[h!]
  		\centering
  		\includegraphics[width=\linewidth]{./figs/Flow_UGV_beacon.png}
  		\caption{UGV beacon tracking}
  		\label{Flow_UGV_beacon}
	\end{figure}
\end{columns}
\end{frame}

\section{ESP32 Based Applications 4}
\begin{frame}
\frametitle{ESP32 Based Applications-4}
\begin{columns}
	\column{0.95\textwidth}
	\begin{figure}[h!]
  		\centering
  		\includegraphics[width=\linewidth]{./figs/Wiring_UAV_ESP32_commlink.png}
  		\caption{UAV Navigation using ESP32 and Android phone}
  		\label{Wiring_UAV_ESP32_commlink}
	\end{figure}
\end{columns}
\end{frame}

\section{Vaman Based Applications 1}
\begin{frame}
\frametitle{Vaman Based Applications-1}
\begin{columns}
	\column{0.75\textwidth}
	\begin{figure}[h!]
  		\centering
  		\includegraphics[width=\linewidth]{./figs/Wiring_UGV_flysky_Vaman.png}
  		\caption{UGV Navigation using Fly-sky transmitter \& receiver (Vaman)}
  		\label{Wiring_UGV_flysky_Vaman}
	\end{figure}
\end{columns}
\end{frame}

\section{Vaman Based Applications 2}
\begin{frame}
\frametitle{Vaman Based Applications-2}
\begin{columns}
	\column{0.75\textwidth}
	\begin{figure}[h!]
  		\centering
  		\includegraphics[width=\linewidth]{./figs/Wiring_UGV_phone_vaman.png}
  		\caption{UGV Navigation using Android phone (Vaman)}
  		\label{Wiring_UGV_phone_vaman}
	\end{figure}
\end{columns}
\end{frame}

\section{SATCOM for UAV Communication}
\begin{frame}{SATCOM for UAV Communication}
    \begin{itemize}
    \item 
    When the cricket ball travel through the air towards the batsman, the air flow around the ball can be \textbf{laminar} or \textbf{turbulent}.
  \end{itemize}
\end{frame} 

\section{Conclusion and Future Directions}
\begin{frame}{Conclusion and Future Directions}
    
  
\end{frame} 





%\begin{figure}[h!]
%  \centering
%  \begin{subfigure}[b]{0.5\linewidth}
%    \includegraphics[width=\linewidth]{./figs/my_8.png}
%%    \caption{Coffee.}
%  \end{subfigure}
%  \begin{subfigure}[b]{0.5\linewidth}
%    \includegraphics[width=\linewidth]{./figs/my_9.png}
%%    \caption{More coffee.}
%  \end{subfigure}
%  \caption{Force coefficient scenarios for trajectory calculations}
%  \label{fig:side3}
%\end{figure}
%\end{frame}



%\begin{frame}
%
%\begin{columns}
%\column{.5\textwidth}
%  \begin{itemize}
%  \item First item.
%  \item Second item.
%  \item Third item.
%  \end{itemize}
%\setbeamertemplate{itemize items}[square]
%  \begin{itemize}
%  \item First item.
%  \item Second item.
%  \item Third item.
%  \end{itemize}
%\column{.5\textwidth}
%\setbeamertemplate{itemize items}[circle]
%  \begin{itemize}
%  \item First item.
%  \item Second item.
%  \item Third item.
%  \end{itemize}
%\setbeamertemplate{itemize items}[ball]
%  \begin{itemize}
%  \item First item.
%  \item Second item.
%  \item Third item.
%  \end{itemize}
%\end{columns}
%
%\end{frame}

\newpage
\bibliography{ref}

\end{document}

