\documentclass[a4paper,twoside]{iiththesis}
\usepackage{graphicx}
\usepackage{caption} 
\captionsetup[table]{skip=10pt}
\usepackage{lscape}
\usepackage{float}
\usepackage{tcolorbox}
\usepackage{hyperref}
\book{Thesis}
\title{Autonomous navigation of UGV/UAV}
\degree{M.Tech. in Communication \& Signal Processing}
\department{Electrical Engineering}
\submitted{June 2022}
\author{Sachinkumar Omprakash Dubey}
\adviser{Dr. G V V Sharma}
\addradviser{Dept. of Electrical Engineering\\ IIT Hyderabad}
%\chair{---------}
%\addrchair{Dept. of Mech Eng \\ IITH}
%\external{----------}
%\addrexternal{Dept. of Chem Eng \\ IITM}
%\internal{----------}
%\addrinternal{Dept. Math \\ IITH}
%\coguide{----------}
\addrcoguide{Dept. of Chem Eng \\ IITH}
\abstract{

}
\acknowledgements{.}
\dedication{.}
\renewcommand{\bibname}{References}
%-------------------------------------------------------
\begin{document}
\tableofcontents
%




\chapter{Citation}
\section{Single citation}
The cite command can be used to create any reference~\cite{Achenbach1995}. i.e. 
\begin{verbatim}
\cite{bibtex_key}
\end{verbatim}


\section{Multiple citation}
You can also cite multiple references using the cite option~\cite{Achenbach1995,Aguiar2004}.. i.e
\begin{verbatim}
\cite{bibtex_key1, bibtex_key2}
\end{verbatim}

Books and Thesis may be cited in the same way~\cite{Bard2001,Iordanidis2002}. The student need not to worry about difference in citation style for journal article, conference, books, thesis etc. This is taken care by bibliography style-file iiththesis.bbl. You are strongly recommended to use $ \backslash $ bibliography{•} rather than individual bibtex entries. By using $ \backslash $bibliography{•} you will never have references which are not cited in the text. You can use any reference manager to create your collection of bibliography.bib. For instance JabRef and Mendeley are reference managers which are freely available.

\chapter{Figures}
\section{Referencing figures}
The figure where ever possible must be centered. Each figure must have a caption centered to the figure. Every single figure in the document must be referred in the text. For example IITH logo is displayed in Fig.~\ref{iithlogo}.

\begin{figure}[h]
\centering
\includegraphics[scale=0.5]{logo}
\caption{This is IITH logo}
\label{iithlogo}
\end{figure}

Use ``Fig". to refer to a figure if the reference to it appears not at the beginning of a sentence. If the sentence starts with reference to figure use ``Figure". For instance refer to the following text.
Figure~\ref{iithlogo} is a compressed logo of IITH.\\

\section{File formats}
You can use jpeg, png, pdf, or eps file format for the figures. However, depending on the file type you will have to use either \textit{pdflatex} or \textit{latex}. Please refer to Chp.~\ref{compiling} for further details.


\chapter{Tables}

\section{Referencing tables}
The tables where ever possible must be centered. The table caption must appear at the top of the table and must be centered to the table. Every table in the document must be referred in the text. Please use capitalized ``T" whenever a reference to table is made. i.e Table~\ref{extable} rather than table~\ref{extable}.
\begin{table}[h]
\centering
\caption{This is an example table.}
\begin{tabular}{l l}
\hline
Parameter & Value \\
\hline
Density & 1 \\
Specific heat & 1 \\
\hline
\end{tabular}
\label{extable}
\end{table}

\chapter{Compiling the \textit{.tex} file }
\label{compiling}

\section{Options}
If you are using jpeg or pdf format for the figures please use\textit{pdflatex} to compile the tex file. If you are using eps format you can use \textit{latex} command to compile the tex file. The \textit{latex} command will create \textit{dvi} output which may be converted to \textit{pdf} by using \textit{dvipdf} on any linux distribution. 

\section{Compilation sequence}
You have to execute the following sequence to commands to get the proper output file.
\begin{verbatim}
latex thesis.tex
bibtex thesis
latex thesis.tex
latex thesis.tex
\end{verbatim}
Notice that you have to tex the document twice after running bibtex.\\

\clearpage
\newpage
%\addcontentsline{toc}{chapter}{References} % Please do not remove this
\bibliographystyle{iiththesis}
\bibliography{references}
\end{document}
